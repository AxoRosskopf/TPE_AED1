\documentclass[a4paper]{article}

\setlength{\parskip}{2mm}
\newcommand{\tab}{~ \qquad}
\input{Algo1Macros}
\usepackage{caratula} % Version modificada para usar las macros de algo1 de ~> https://github.com/bcardiff/dc-tex

\begin{document}

\titulo{Trabajo Practico de Especificación}
\subtitulo{Buscaminas}
\fecha{23 de septiembre de 2022}
\materia{Algoritmos y Estructuras de Datos I}
\grupo{Grupo martinlau}

\newcommand{\pos}{pos}
\newcommand{\tablero}{tablero}
\newcommand{\jugadas}{jugadas}
\newcommand{\banderitas}{banderitas}


% Pongan cuantos integrantes quieran
\integrante{Apellidos, Lautaro}{370/22}{lautisantil@gmail.com }
\integrante{Apellidos, Sofía}{1654/21 }{sofiapotter07@gmail.com }
\integrante{Campoverde, Axel}{258/22}{a.i.cpvd3000@gmail.com }
\integrante{Menalled, Martín}{339/11}{martinmenalled@gmail.com}

\maketitle


\section{Definición de Tipos}

\begin{description}
    \item type \pos
    \item type \tablero
    \item type \jugadas
    \item type \banderitas
    
\end{description}

\section{Constantes}

% AUXILIARES USADOS 

\aux{cantidadDeMinasTablero}{ t : \tablero}{\ent}{\sum_{i = 0}^{|t|-1}\sum_{j = 0}^{|t|- 1} 
    $if (t[i][j] = \True ) then 1 else 0 $ fi}
    
\aux{convierteDosEnterosEnUnaPos}{x: \ent, y:\ent}{pos}{(x,y)}{}

% PREDICADOS USADOS

\pred{esAdyacente}{X: \pos,  P:\pos}{(\longitud{X_0-P_0}=1 \wedge (\longitud{X_1-P_1}=1  \vee x_1=p_1 ) \bigvee (X_0=P_0 \wedge \longitud{X_1-P_1}=1 )}


\pred{juegoValido}{ t : \tablero , j : \jugadas}{}{}

\pred{predPerdio}{t : \tablero, j : \jugadas}{
   		cantidadDeMinas( t , j) = 1
    }
\pred{tableroValido}{t : \tablero}{
		(\forall i : \ent)( (0 \leq i < |t|) \y (|t| = |t[0]|))\implicaLuego |t[i]| = |t[0]|
}

\section{Problemas}

% RESOLUCION DE LA PARTE 1

\subsection*{Parte 1: Juego Básico}


\subsection*{Ejercicio 1}

\aux{minasAdyacentes}{t:\tablero, p: \pos}{\ent}{\sum_{i=0}^{\longitud{t}-1}\sum_{j=0}^{\longitud{t}-1} \IfThenElse{esAdyacente(convierteDosEnterosEnUnaPos (i, j), p) \wedge t[i][j]}{1}{0}} 
{\color{red} estaba mal como pasabamos la posici[on del tablero porque asi como estaba era un bool. Armé una aux que le damos la i y la j y nos arme una tupla pos, revisen si esta ok.}


{\color{blue} Mas o menos entendi lo que hiciste, se me hace raro que de un numero(i , j)le pasas a una tupla para que luego agarres de nuevo estos  mismos numeros que ahora estan en tupla y los compares con pos, no terminas por hacer un paso de mas?? porque valdria esto por ejemplo 
$|i - p_{0}| = 1$}

\subsection{Ejercicio 4}

\begin{proc}{perdio}{\In i: \tablero, \In j: \jugadas, \Out res: \bool}{}
	\pre{  juegoValido(t , j) \y tableroValido(t) }
    \post{ res : \True \leftrightarrow predPerdio(t , j)}
\end{proc}



\subsection{Ejercicio 5}
\begin{proc}{gano}{\In i: \tablero, \In j: \jugadas, \Out res: \bool}{}
	\pre{  juegoValido(t , j) \y tableroValido(t) \y predPerdio(t , j) }
    \post{ res : \True \leftrightarrow predGano(t , j)}
\end{proc}


\subsection{Ejercicio 6}
\begin{proc}{jugar}{\In t : \tablero , \In b : \banderitas, \In p : \pos , \Inout j : \jugadas}{}
    \pre { tableroValido(t) \y juegoValido(t , j) \y \neg predPerdio(t , j) \y \neg predGano(t , j) \y \neg posEnJugadas(p , j) \\ \y \neg posEnBanderitas(p , b) \y j = j_{0}}
    \post{(\forall i : \ent)(0 \leq i < |j_{0}|) \implicaLuego((j_{o}[i] \in j) \y ((\exists k : \ent)(0 \leq k < |j|) \yLuego (j[k][0]_{0} = p_{0} \y j[k][0]_{1} = p_{1} \y j[k][1] = minasAdyacentes(t , j))))}
\end{proc}


% FINALIZA PARTE 1 COMIENZA PARTE 2


\subsection*{Parte 2: Despejar los vacıos}

\subsection{Ejercicio 7}


\pred{caminoLibre}{t : \tablero,p_{0} : \pos ,p_{1} : \pos}{ tableroValido(t) \y posicionValida(p_{0}) \y posicionValida(p_{1})\y ((\exists s : \TLista{\pos})((\forall i : \ent)(0 \leq i < |s|-1)\implicaLuego (minasAdyacentes(t, s[1]) = 0 \y esAdyacente(s[i] ,s[i+1])) \yLuego p_{0} = s[0] \y p_{1} = s[|s| - 1]))}


\end{document}

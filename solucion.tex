\documentclass[a4paper]{article}

\setlength{\parskip}{2mm}
\newcommand{\tab}{~ \qquad}
\input{Algo1Macros}
\usepackage{caratula} % Version modificada para usar las macros de algo1 de ~> https://github.com/bcardiff/dc-tex
\usepackage{xcolor}


\begin{document}

\titulo{TP de Especificación}
\subtitulo{}
\fecha{23 de septiembre de 2022}
\materia{Algoritmos y Estructuras de Datos I}
\grupo{Grupo martinlau}

\newcommand{\dato}{\textit{Dato}}
\newcommand{\individuo}{\textit{Individuo}}

% Pongan cuantos integrantes quieran
\integrante{Apellidos, Lautaro}{370/22}{lautisantil@gmail.com }
\integrante{Apellidos, Sofía}{1654/21 }{sofiapotter07@gmail.com }
\integrante{Apellidos, Axel}{258/22}{a.i.cpvd3000@gmail.com }
\integrante{Menalled, Martín}{339/11}{martinmenalled@gmail.com}

\maketitle


\section{Definición de Tipos}
\begin{description}
	\item type \dato = \ent \hspace{1cm}
	\item type \individuo = $\TLista{\dato}$
\end{description}

{\color{red}FALTA ARMAR ESTO}

\section{Constantes}
	\aux{MIN}{}{\ent}{1}
	\aux{MAX}{}{\ent}{10}

{\color{red}??? ESTO?? QuE PONEMOS ACA?}

\section{Problemas}
\subsection*{Parte 1: Juego Básico}
\subsection*{Ejercicio 1}

\aux{minasAdyacentes}{t:tablero, p: pos}{\ent}{\sum_{i=0}^{\longitud{t}-1}\sum_{j=0}^{\longitud{t}-1} \IfThenElse{esAdyacente(convierteDosEnterosEnUnaPos (i, j), p) \wedge t[i][j]}{1}{0}} 
{\color{red} estaba mal como pasabamos la posici[on del tablero porque asi como estaba era un bool. Armé una aux que le damos la i y la j y nos arme una tupla pos, revisen si esta ok.}


\pred{esAdyacente}{X: pos,  P:pos}{(\longitud{X_0-P_0}=1 \wedge (\longitud{X_1-P_1}=1  \vee x_1=p_1 ) \bigvee (X_0=P_0 \wedge \longitud{X_1-P_1}=1 )}

\aux{convierteDosEnterosEnUnaPos}{x: \ent, y:\ent}{pos}{(x,y)}{}

\end{document}
